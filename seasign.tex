\documentclass{llncs}

%\usepackage{amsmath,amssymb}
\usepackage{amsmath,amssymb,fullpage}
\usepackage{times}
\usepackage{algorithm,algorithmicx,algpseudocode}
\usepackage[utf8]{inputenc}
\usepackage[OT1]{fontenc}

\newcommand{\C}{\mathcal{C}}
\newcommand{\D}{\mathcal{D}}
\newcommand{\E}{\mathcal{E}}
\newcommand{\F}{\mathbb{F}}
\newcommand{\Fpbar}{\overline{\mathbb{F}}_p}
\newcommand{\Fqbar}{\overline{\mathbb{F}}_q}
\newcommand{\FF}{\mathcal{F}}
\newcommand{\N}{\mathbb{N}}
\newcommand{\OO}{\mathcal{O}}
\newcommand{\Q}{\mathbb{Q}}
\newcommand{\R}{\mathbb{R}}
\newcommand{\Z}{\mathbb{Z}}
\newcommand{\ch}{\text{ch}}
\newcommand{\End}{\text{End}}
\newcommand{\Cl}{\text{Cl}}
\newcommand{\seed}{\textsf{seed}}

\renewcommand{\a}{\mathfrak{a}}
\renewcommand{\b}{\mathfrak{b}}
\let\cedil\c
\renewcommand{\c}{\mathfrak{c}}
\renewcommand{\l}{\mathfrak{l}}
\newcommand{\e}{\textbf{e}}
\newcommand{\f}{\textbf{f}}
\newcommand{\x}{\textbf{x}}
\newcommand{\z}{\textbf{z}}

% THEOREM ENVIRONMENTS
%\newtheorem{definition}{Definition}
%\newtheorem{example}{Example}
%\newtheorem{theorem}{Theorem}



\title{SeaSign}

\author{Luca De Feo \and Steven D. Galbraith}
\institute{Mathematics Department, University of Auckland, NZ.
\email{s.galbraith@auckland.ac.nz}}



\date{\today}


\begin{document}
\pagestyle{plain}

\maketitle


\begin{abstract}

\end{abstract}



\section{Intro}

Stolbunov~\cite{Sto12} was the first to give a signature scheme based on isogeny problems.
Due to renewed interest in this problem, especially CSIDH~\cite{CLMPR18} and the scheme by De Feo et al~\cite{DFKS18}.

For $B \in \N$ we use the notation $[-B,B]$ for the set of integers $u$ with $-B \le u \le B$.



Currently it is a major problem to get practical signatures from isogeny problems.
Yoo et al (see Table~1 of~\cite{YAJJS17}) state signatures of over 100 kilo-bytes.
This can be reduced a small amount using some optimisations (e.g., see~\cite{GPS17}), but will still be in the tens of kilobytes.
In contrast, in this paper we should be able to get signatures smaller than a kilobyte, which is better than lattice signatures!!!
Unfortunately, signing and verification are very slow, but we might just about be able to live with that in certain applications.




\section{Background}

Notation:
\begin{itemize}
\item For an elliptic curve $E$ over a field $K$ we define $\End(E)$ to the the ring of endomorphisms of $E$ defined over the algebraic closure of $K$, and $\End_K(E)$ to be the the ring of endomorphisms defined over $K$.
\item Given two $\OO$-ideals $\a, \b$ we write $\a \cong \b$ if $\a$ and $\b$ are equivalent (meaning that $\a \b^{-1}$ is a principal fractional $\OO$-ideal). 
\end{itemize}


Let $p$ be a prime.
Let $E$ be an ordinary elliptic curve over $\F_p$ with $\End(E) \cong \OO$ or $E$ a supersingular curve over $\F_p$ with $\End_{\F_p}(E) \cong \OO$ where $\OO$ is an order in an imaginary quadratic field.
Let $\Cl(\OO )$ be the ideal class group of $\OO$.
One can define the action of an $\OO$-ideal $\a$ on the curve $E$ as the image curve $E'$ under the isogeny $\phi : E \to E'$ whose kernel is equal to the kernel ideal $E[ \a ] = \{ P \in E( \Fpbar ) : \alpha(P) = 0 \; \forall \alpha \in \a \}$.
We denote $E'$ by $\a * E$.

The set $\{ j(E) \}$ of isomorphism classes of elliptic curves with $\End(E) \cong \OO$ is a principal homogenous space for $\Cl(\OO )$.
General references for all this are Couveignes~\cite{Couv06}
and Stolbunov~\cite{Sto12}.

Given a generic ideal $\a\subset\OO$, the best known algorithm to compute $\a*E$ has subexponential complexity in $\log(\#\Cl(\OO))$~\cite{JS10}.
To make this action efficient one must work instead with ideals $\a = \prod_{i=1}^n \l_i^{e_i}$ where $\l_1, \dots, \l_n$ are prime $\OO$-ideals of small norm $\ell_i$ and where $(e_1, \dots, e_n)$ is an appropriately chosen vector of exponents.
Then, the action of $\a$ can be computed as a composition of isogenies of degree $\ell_i$.
Throughout the paper we assume that $\{ \l_1, \dots, \l_n \}$ is a set of non-principal prime ideals in $\OO$, generating $\Cl(\OO)$, of norm polynomial in the size of the class group.
Theoretically we have the bounds $\#\Cl(\OO) = O( \sqrt{p} \log(p) )$ and, assuming a generalised Riemann hypothesis, $\ell_i = O( \log(p)^2 )$.
% NdL: Is #Cl(O) = O(..) really what we want to say? ≈ seems more appropriate
In practice one usually takes $\ell_i=O(\log(p))$ for efficiency reasons; heuristically, this is more than enough to generate the class group.

The basic computational assumption is the ideal action problem:

\begin{definition}\label{defn:ass1} (Ideal action problem -- better name???)
Given two elliptic curves $E$ and $E_A$ over the same field with $\End(E) = \End(E_A) = \OO$. Find an ideal $\a$ such that $j( E_A ) = j( \a * E )$.
\end{definition}

The best classical algorithms for this problem have exponential time, but there are subexponential-time quantum algorithms for it~\cite{Kup,regev04,childs2014constructing,Kuperberg2013} (Kuperberg, Regev, Childs-Jao-Soukharev, Bonnetain- Schrottenloher (eprint 2018/537), Biasse-Iezzi-Jacobson (arXiv:1806.03656)).

Note that this problem admits a random self-reduction: given an instance $(E, E_A)$ one can choose random ideal classes $\b_1, \b_2$ and construct the instance $(E_1, E_2) = (\b_1 * E, \b_2 * E_A )$, which is now uniformly distributed across the set of pairs of isomorphism classes of curves in the isogeny class.
If $\a'$ is a solution to the instance $(E_1, E_2)$ then any ideal equivalent to the fractional ideal $\a'\b_1 \b_2^{-1}$ is a solution to the original instance.

When instantiating the group action in practice, one must choose parameters that make evaluating isogenies of degree $\ell_i$ as efficient as possible.
In the ordinary case, De Feo, Kieffer and Smith~\cite{DFKS18} introduced a method to use the more efficient V\'elu's formulas~\cite{velu71} for some primes $\ell_i$, but were unable to generalize it for sufficiently many.
By using supersingular curves over a field $\F_p$ with $p+1 = 4\prod_{i=1}^n\ell_i$, CSIDH~\cite{CLMPR18} manages to apply V\'elu's formulas to all primes $\ell_i$.
For key exchange, CSIDH samples the exponent vectors $\e = (e_1, \dots, e_n) \in [-B,B]^n \subseteq \Z^n$ for a suitable constant $B$.
% such that $(2B+1)^n\ge\#\Cl(\OO)\approx p\log p$; this ensures that, heuristically, the key space covers all of $\Cl(\OO)$.


An issue that arises in our security arguments is whether or not we are sampling ideal classes uniformly during key generation and signing (for precise details see Definition~\ref{defn:sampling-distributions} and the discussion that follows it).
Recall there is a straightforward meet-in-the-middle attack when given $E$ and $\a * E$ for $\a = \prod_{i=1}^n \l_i^{e_i}$ over $e_i \in [-B, B]$.
We compute lists (assume $n$ is even)
\[
   L_1 = \{ ( \prod_{i=1}^{n/2} \l_i^{e_i} ) * E : e_i \in [-B,B] \} \text{ and }
   L_2 = \{ ( \prod_{i=n/2 + 1}^{n} \l_i^{e_i} ) * E_A : e_i \in [-B,B] \}.
\]
If $L_1 \cap L_2 \ne \empty$ then we have solved the isogeny problem.
This attack can be used even when the set of ideal classes generated is a small subset of $\Cl( \OO )$.
Hence for security we require $(2B+1)^n > 2^{2 \lambda}$, where $\lambda$ is the security parameter.

(** CHECK **) There is a quantum variant of this algorithm using Tani, which means we need $3 \lambda$ bits to have post-quantum security.

Note that Kuperberg's algorithm uses the entire class group, and so there seems to be no improvement to the subexponential attack even if we only sample a subset of the class group, or if the sampling of ideal classes is biased from the uniform distribution.

OPEN Q: Can Kuperberg be done if we sample from a subset of $\Cl(\OO)$????




By taking into account the best known attacks, the CSIDH authors propose parameters for the three NIST categories~\cite{NIST2016}, as summarized in Table~\ref{tab:csidh-parms}.
Their implementation of the smallest parameter size CSIDH-1 computes one class group action in under 100ms on a 3.5GHz processor.


\begin{table}
  \centering
  \begin{tabular}{l | r | r | r | r | r | r | r | r}
    & $n$ & $\log_2 p$ & $B$ & NIST level & classical security & quantum security & message size & private key size \\
    \hline
    CSIDH-1 &  74 &  500 &  5 & 1 & 125 bits &  61 qbits &  63B &  32B\\
    CSIDH-3 & 131 & 1020 &  7 & 3 & 255 bits &  93 qbits & 128B &  64B\\
    CSIDH-5 & 208 & 1787 & 10 & 5 & 447 bits & 129 qbits & 224B & 115B
  \end{tabular}
  \caption{Proposed parameters for CSIDH~\cite{CLMPR18}.  Effective
    parameters $p$, $n$ and $B$ for CSIDH-3 and CSIDH-5 were not given
    in the paper, and are produced here following their methodology.
    Message size is the number of bytes to represent a $j$-invariant, and private key size is the space required to store the exponent vector $\e \in \Z^n$.}
  \label{tab:csidh-parms}
\end{table}



\subsection{Identification protocols and signature schemes}

Do we need to recall the definitions of id-protocols and sig schemes.

There are various transforms to turn an ID protocol into a sig scheme for classical or post-quantum security~\cite{AABN02,DFG13,GCZ16,Katz10,Un15,Un17}.
We focus in this paper on classical adversaries and therefore use the Fiat-Shamir scheme.




\section{Basic Signature Scheme}\label{sec:basic-scheme}

Section 2.B of Stolbunov's PhD thesis~\cite{Sto12} contains a sketch of a signature scheme based on isogeny problems (though his decription is not complete and he does not give a proof of security).
It is a Fiat-Shamir scheme based on an identification protocol.
Section 4 of Couveignes~\cite{Couv06} also sketches the identification protocol, but does not mention that it can be converted to a signature scheme.


The public key consists of $E$ and $E_A = \a * E$, where $\a = \prod_{i=1}^n \l_i^{e_i}$ is the private key.
To construct the private key one uniformly chooses an exponent vector $\e = (e_1, \dots, e_n) \in [-B,B]^n \subseteq \Z^n$ for some suitably chosen constant $B$.

In the identification protocol the prover generates $t$ random ideals $\b_k = \prod_{i=1}^n \l_i^{f_{k,i}}$ for $1 \le k \le t$ and computes $E_k = \b_k * E$.
Here the exponent vectors $\f_k = ( f_{k,1}, \dots, f_{k,n} )$ are uniformly and independently sampled in $[-B,B]^n$.
The prover sends $(j( E_k ) : 1 \le k \le t )$ to the verifier.
The verifier responds with $t$ uniformly chosen challenge bits $b_1, \dots, b_t \in \{0,1\}$.
If $b_k = 0$ the prover responds with $\f_k = ( f_{k,1}, \dots, f_{k,n} )$ and the verifier checks that $j(E_k) = j( (\prod_{i=1}^n \l_i^{f_{k,i}}) * E )$.
If $b_k = 1$ the prover responds with a representation of $\b_k \a^{-1}$. Stolbunov notes that sending the vector $\f_k - \e = (f_{k,i} - e_k )$ would not be secure as it would leak the private key; but neither he nor Couveignes explains how to prevent this leakage.
When $b_k=1$ the verifier checks that $j(E_k) = j( (\b_k \a^{-1}) * E_A )$.
A cheating prover (who does not know the private key) can succeed with probability $1/2^t$.

To obtain a signature scheme one applies the Fiat-Shamir transform, and hence obtains the challenge bits $b_k$ as the hash value $H( j(E_1), \dots, j(E_t) , m )$ where $H$ is a cryptographic hash function with $t$-bit output and $m$ is the message to be signed.
The signature consists of the binary string $b_1\cdots b_t$ and the representations of the ideal classes $\b_k$ when $b_k = 0$ and $\b_k \a^{-1}$ when $b_k = 1$.

The verifier computes, for $1 \le k \le t$, $E_k = \b_k * E$ when $b_k = 0$ and $E_k = \b_k \a^{-1} * E_A$ when $b_k = 1$. The verifier then computes $H( j( E_1), \dots, j(E_t), m )$ and checks whether this is equal to the binary string $b_1\cdots b_t$, and accepts the signature if and only if the strings agree.





In this paper we propose two solutions to the problem of representing $\b_k \a^{-1}$ without leaking the private key.
Our main solution uses ideas from lattice cryptography and we describe this in the remainder of the section.
An alternative solution requires knowledge of a relation lattice in the ideal class group, which can be computed efficiently using a quantum computer. We describe this alternative solution in an appendix.





\subsection{Using rejection sampling}\label{sec:sig-reject-sample}

% The approach in Section~\ref{sec:sig-relation-lattice} is theoretically elegant, but the assumption that the relation lattice $L$ can be computed is doubtful for practical post-quantum signatures.
%Hence this section contains our main solution to the problem of representing the ideal class $\b_k \a^{-1}$.
The idea is to use rejection sampling in exactly the way proposed by Lyubashevsky~\cite{Lyu09} in the context of lattice signatures.

Let $B > 0$ be a constant. When generating the private key we sample uniformly $e_i \in [-B, B]$ for $1 \le i \le n$. Let $\e = ( e_1, \dots, e_n )$.
The value $B$ is presumably chosen large enough that $\prod_{i=1}^n \l_i^{e_i}$ covers most ideal classes and so that the output distribution is close to uniformly distributed in $\Cl(\OO)$, but we try to avoid any explicit requirement or assumption that this distribution is uniform.

The idea is to sample the exponents $f_{k,i}$ uniformly in $[-(nt+1)B, (nt+1)B]$, where $t$ is the number of parallel rounds of the identification protocol and $n$ is the number of primes.
Let $\f_k = (f_{k,1}, \dots, f_{k,n} )$.
It is certainly the case that $\b_k = \prod_{i=1}^n \l_i^{f_{k,i}}$ is likely to be close to uniformly distributed, but we shouldn't need to make any assumption about it.

If the $k$-th challenge bit $b_k$ is zero then the prover responds with $\f_k = ( f_{k,1}, \dots, f_{k,n} )$ and the verifier checks that $j(E_k) = j( (\prod_{i=1}^n \l_i^{f_{k,i}}) * E )$ as in the basic scheme above.\footnote{In the scheme and analysis I actually apply rejection sampling to the case $b_k = 0$. It doesn't really matter one way or the other.}
If $b_k = 1$ then the prover is required to provide a representation of $\b_k \a^{-1}$, the idea is to compute the vector $\z_k = (z_{k,1}, \dots, z_{k,n}) $ defined by $z_{k,i} = f_{k,i} - e_i $ for $1 \le i \le n$.
As already noted, outputting $\z$ directly would potentially leak the secret.
To prevent this leakage we only output $\z_k$ if all its entries satisfy $| z_{k,i} | \le nt B$.
We give the signature scheme in Figure~\ref{fig:sig-scheme}.
It remains to show that in the accepting case the vector leaks no information about the private key, and that the rejecting case occurs with low probability. We do this in the following two lemmas.

\begin{lemma} \label{lem:sim2}
The distribution of vectors $\z_k$ output by the signing algorithm is the uniform distribution and therefore is independent of the private key $\e$.
\end{lemma}

\begin{proof}
Let $U = [-(nt+1)B, (nt+1)B]$. Then $\#U = 2(nt + 1)B + 1$.
If $e \in [-B, B]$ then 
\[
    [-ntB, ntB] \subseteq  U - e = \{ f - e : f \in U \} \subseteq [-(nt+2)B, (nt+2)B ].
\]
Hence, when rejection sampling (only outputting values $f_{k,i} - e_i$ in the range $[-ntB, ntB]$) is applied then the output distribution of $\z_k$ is the uniform distribution $[-ntB, ntB]^n$.
This argument does not depend on the choice of $\e$, so the output distribution is independent of $\e$.
\end{proof}



\begin{lemma}
The probability that the signing algorithm does not output $\bot$ is at least $1/e > 1/3$.
\end{lemma}

\begin{proof}
Let notation be as in the proof of Lemma~\ref{lem:sim2}.
For fixed $e \in [-B, B]$ and uniformly sampled $f \in U$, the probability that a value $f-e$ lies in $[-ntB, ntB]$ is
\[
   \frac{2ntB + 1}{2(nt+1)B + 1}  = 1 - \frac{2B}{2(nt+1)B + 1} \ge 1 - \frac{1}{nt+1}.
\]
Hence, the probability that all of the values $z_{k,i}$ over $1 \le k \le t, 1 \le i \le n$ lie in $[-ntB, ntB]$ is at least $(1 - 1/(nt+1))^{nt}$.
Using the inequality $1 - 1/(nt+1) \ge e^{-1/nt}$ for $nt \ge 1$ it follows that the probability that all values are in the desired range is at least
\[
   \left( e^{-1/nt} \right)^{nt} = e^{-1}.
\]
This completes the proof.
\end{proof}


We can therefore get a rough idea of parameters and efficiency for the scheme.
For security we need a large enough set of private keys. So we need $(2B+1)^n$ large enough.
We need $t$ at least 128 so that an attacker cannot guess the hash value or invert the hash function.
The vectors $\z_k$ have entries of size bounded by $ntB$, which means that for each $k$ and each prime $\l_i$ one needs to compute up to $ntB$ isogenies of degree $\ell_i$.
This means that the total number of isogeny computations is upper bounded by $(nt)^2 B$.
The quadratic dependence on $nt$ is going to be a big pain in practice.
For example, taking $t=128, n = 50, B = 6$ gives around $2^{27}$ isogeny computations in verification!
We can make $t$ small using the techniques in later sections, but one needs $n$ large unless $B$ is going to get very large. So even going down to $t=8$ still has signatures requiring around $2^{20}$ isogeny computations.

It might be worth to consider different shaped boxes (i.e., favouring smaller prime isogeny degrees) like done by Stolbunov. But it won't bypass the $nt$ issue

%\begin{figure} 



%If a challenge bit $b_k = 1$ and the prover is required to provide a representation of $\b_k \a^{-1}$, the idea is to compute the vector $\z = (z_1, \dots, z_n) $ so that $z_k = f_{k,i} - e_k $.
%Now, if all $|z_k| \le tB$ then we accept $\z$ as a ``safe'' representation of the ideal, otherwise we reject $\z$ and the protocol fails.
%
%f $E$ and $E_A = \a * E$, where $\a = $ is the private key.
%To construct the private key one unif
%
% $1 \le k \le t$ and computes $E_k = \b_k * E$.
%Here the exponent vectors $\f_k = ( f_{k,1}, \dots, f_{k,n} )$ are uniformly and independently sampled in $[-B,B]^n$.

%\begin{figure}[!htb]
\begin{figure}
\begin{minipage}{.45\textwidth}
\begin{algorithm}[H]
	\caption{KeyGen \label{alg:KeyGen}}
%	\caption{KeyGen}
	\textbf{Input:} $B$, $\l_1, \dots. \l_n$, $E$

	\textbf{Output:} $sk =\e$ and $pk = E_A$

	\begin{algorithmic}[1]
		\State $\e \leftarrow [-B,B]^n$ 
		\State $E_A = ( \prod_{i=1}^n \l_i^{e_i} ) * E$
		\State \Return $sk= \e$, $pk = E_A$
	\end{algorithmic}
\end{algorithm}

\vskip -1.2cm

\begin{algorithm}[H]
	\caption{Sign \label{alg:Sign}}
	\textbf{Input:} $\text{message}$, $(E,E_A)$, $\e$

	\textbf{Output:} $(\z_1, \dots, \z_t)$, $(b_1 , \dots, b_t)$

	\begin{algorithmic}[1]
		\For {$k=1 , \dots , t$}
		\State $\f_k \leftarrow [-(nt+1)B,(nt+1)B]^n$ 
		\State $E_k = ( \prod_{i=1}^n \l_i^{f_{k,i}} ) * E$
		\EndFor
		\State $b_1 \Vert \cdots \Vert b_t = H( j(E_1) , \dots, j(E_t), \text{message} )$
		\For {$k=1, \dots, t$}
		\If{$b_k=0$}
		\State $\z_k = \f_k$
		\Else
		\State $\z_k = \f_k - \e$
		\EndIf
		\If{$\z_{k} \not\in [-ntB,ntB]^n$} \State \Return $\bot$ \EndIf
		\EndFor
		\State \Return $\sigma = (\z_1, \dots, \z_t, b_1 , \dots, b_t)$
	\end{algorithmic}
\end{algorithm}
\end{minipage}
 \ \ \ \ \ \ \ \ \ \ \ \ 
\begin{minipage}{0.45\textwidth}
\begin{algorithm}[H]
	\caption{Verify \label{alg:Verify}}
	\textbf{Input:} $\text{message}$, $(E,E_A)$, $\sigma$

	\textbf{Output:} Valid/Invalid

	\begin{algorithmic}[1]
		\State Parse $\sigma$ as $(\z_1, \dots, \z_t, b_1 , \dots, b_t)$
		\For {$k=1 , \dots , t$}
		\If{$b_k=0$}
		\State $E_k = ( \prod_{i=1}^n \l_i^{z_{k,i}} ) * E$
		\Else
		\State $E_k = ( \prod_{i=1}^n \l_i^{z_{k,i}} ) * E_A$
		\EndIf
		\EndFor
		\State $b_1' \Vert \cdots \Vert b_t' = H( j(E_1) , \dots, j(E_t), \text{message} )$
		\If{$(b_1', \dots, b_t') = (b_1 , \dots, b_t)$} \State \Return Valid
		\Else \State \Return Invalid \EndIf
	\end{algorithmic}
\end{algorithm}
\end{minipage}
\caption{The basic signature scheme using rejection sampling.\label{fig:sig-scheme}}
\end{figure}




\subsection{Security proof}
\label{sec:security-proof}

We now prove security of the basic scheme in the random oracle model against a classical adversary. 
%The proof covers both cases.
The proof technique is the standard approach that uses the forking lemma.
We do not consider quantum adversaries, or give a proof in the quantum random oracle model.

First we need to discuss some subtleties about the distribution of ideal classes coming from the key generation and signing algorithms.

\begin{definition} \label{defn:sampling-distributions}
Fix distinct ideals $\l_1, \dots, \l_n$.
For $B \in \N$ define $\D_B$ to be the distribution on $\Cl( \OO )$ corresponding to the ideal class of $\prod_{i=1}^n \l_i^{e_i}$ over uniformly sampling $\e \in [-B,B]^n$.
Define $M_B$ to be an upper bound on the probability, over $\a, \b$ sampled from $\D_B$, that $\a \cong \b$
\end{definition}

In other words, $\D_B$ is the output distribution of the public key generation algorithm.
Understanding the distribution $\D_B$ is non-trivial in general.
For small $B$ and $n$ (so that $(2B+1)^n \ll \#\Cl(\OO)$) we expect $\D_B$ to be the uniform distribution on a subset of $\Cl(\OO)$ of size $(2B+1)^n$. For fixed $n$ and large enough $B$ it should be the case that $\D_B$ is very close to the uniform distribution on $\Cl(\OO)$.
A full study of the distribution $\D_B$ is beyond the scope of this paper, but is a good problem for future work.

For the isogeny problem to be hard for public keys we certainly need $M_B \le 1/2^\lambda$, where $\lambda$ is the security parameter.
In the proof we will need to use $M_{ntB}$, since the concern is about the auxiliary curves generated during the signing algorithm. We do not require these curves to be uniformly sampled, but in practice we can certainly assume that $M_{ntB} = O( 1/\sqrt{p} )$. In any case, it is negligible in the security parameter.



\begin{definition} \label{defn:ass1p}
(Restricted ideal action problem ***)
Let notation be as in the key generation protocol of the scheme.
Given $(E, E_A)$, where $E_A = \a * E$ for some ideal $\a = \prod_{i=1}^n \l_i^{e_i}$ and the exponent vector $\e = (e_1, \dots, e_n)$ is uniformly sampled in $[-B,B]^n \subseteq \Z^n$, to compute any ideal equivalent to $\a$.
\end{definition}

Depending on how close to uniform is the distribution $\D_B$, this problem may or may not be equivalent to Definition~\ref{defn:ass1} and may or may not have a random self-reduction.


We recall the forking lemma, in the formulation of Bellare and Neven~\cite{BN06}.

\begin{lemma} \label{forking-lemma} (Bellare and Neven~\cite{BN06})
Fix an integer $Q \ge 1$. Let $A$ be a randomized algorithm that takes as input $h_1, \dots, h_Q \in \{0,1\}^t$ and outputs $(J, \sigma)$ where $ 1\le J \le Q$ with probability $\wp$.
Consider the following experiment: $h_1, \dots, h_Q$ are chosen uniformly at random in $\{0,1\}^t$; $A(h_1, \dots, h_Q )$ returns $(I,\sigma)$ such that $I \ge 1$; $h_I', \dots, h_Q'$ are chosen uniformly at random in $\{0,1\}^t$; $A( h_1, \dots, h_{I-1}, h_I', \dots, h_Q' )$ returns $(I', \sigma')$.
Then the probability that $I' = I$ and $h_{I}' \ne h_I$ is at least $\wp( \wp/Q - 1/2^t )$.
\end{lemma}


\begin{theorem}\label{thm:security-basic}
In the random oracle model, the basic signature scheme is unforgeable under a chosen message attack under the assumption of Definition~\ref{defn:ass1p}.
\end{theorem}

\begin{proof}
Consider a polynomial-time adversary $A$ against the signature scheme. So $A$ takes a public key, makes queries to the hash function $H$ and the signing oracle, and outputs a forgery of a signature for the public key.

Let $(E, E_A = \a * E )$ be an instance of Definition~\ref{defn:ass1p}.
The simulator runs the adversary $A$ with public key $(E, E_A)$.
% (or apply the random self-reduction mentioned earlier) in the random oracle model.

Suppose the adversary $A$ makes at most $Q$ (polynomial in the security parameter) queries in total to either the random oracle $H$ or the signing oracle. We now explain how the simulator responds to these queries. The simulator maintains a list, initially empty, of pairs $(x, H(x))$ for each value of the random oracle that has been defined.

\noindent \textbf{Sign queries:}
To answer a Sign query on message $m$ the simulator chooses 
$t$ uniformly chosen bits $b_1, \dots, b_t \in \{0,1\}$.
When $b_k = 0$ the simulator randomly samples $z_k \leftarrow [-ntB,ntB]^n$ and sets $\b_k = \prod_{i=1}^n \l_i^{z_{k,i}}$ and computes $E_k = \b_k * E$, just like in the real signing algorithm.
When $b_k = 1$ the simulator chooses a random ideal $\c_k = \prod_{i=1}^n \l_i^{z_{k,i}}$ for $z_{k,i} \in [-ntB, ntB]$.
%(consistent with the protocol under consideration; so using Lemma~\ref{lem:sim2}) 
and computes $E_k = \c_k * E_A$.
By Lemma~\ref{lem:sim2}, the values $j( E_k )$ and $\z_k$ are distributed exactly as in the real signing algorithm.
We program the random oracle (update the hash list) so that $H( j( E_1), \dots, j(E_t), m ) := b_1 \cdots b_t$, unless the random oracle has already been defined on this input in which case the simulation fails and outputs $\perp$.
The probability of failure is at most $Q/M_{ntB}^t$, 
where $M_{ntB}$ is defined in Definition~\ref{defn:sampling-distributions} to be an upper bound on the probability of a collision in the sampling of ideal classes.
Note that $Q/M_{ntB}^t$ is negligible.
Assuming the simulation does not fail, the output is a valid signature and is indistinguishable from signatures output by the real scheme in the random oracle model.

\noindent \textbf{Hash queries:}
To answer a random oracle query on input $x$ one checks if $(x,y)$ already appears in the list, and if so returns $y$. Otherwise one chooses uniformly at random $y \in \{0,1\}^t$ and sets $H(x) := y$ and adds $(x,y)$ to the list.

Eventually $A$ outputs a forgery $(m, \z_1, \dots, \z_t, b_1\cdots b_t)$ that passes the verification equation.
Define $\c_k = \prod_i \l_i^{z_{k,i}}$.
The proof now invokes the Forking Lemma (see Bellare-Neven~\cite{BN06}). The adversary is replayed with the same random tape and the exact same simulation, except that one of the hash queries is answered with a different binary string.
With non-negligible probability the adversary outputs a forgery with the same message $m$ and the same input $(j(E_1), \dots, j(E_t), m)$ to $H$, but a different output string $b_1'\cdots b_t'$. Let $k$ be an index such that $b_k \ne b_k'$ (without loss of generality $b_k = 0$ and $b_k' = 1$). Then the ideal classes $\c_k$ and $\c_k'$ in the two signatures are such that $j( \c_k * E ) = j( \c_k' * E_A )$ and so $\c_k' \c_k^{-1}$ is a solution to the problem instance.
\end{proof}


We make two observations about the use of the forking lemma.
First, as always, the proof is not tight since if the adversary succeeds with probability $\epsilon$ then the simulator solves the computational problem with probability proportional to $\epsilon^2$.
Second, the hash output length $t$ in Lemma~\ref{forking-lemma} only appears in the term $1/2^t$, so it suffices to take $t = \lambda$.
There may be situations where a larger hash output is needed; for more discussion about hash output sizes we refer to Neven, Smart and Warinschi~\cite{NSW09}.





\section{Smaller sigs}\label{sec:smaller-sigs}


The signature size of the basic scheme is very large, since the sigma protocol that underlies the identification scheme only has single bit challenges. 
In practice we need $t \ge 128$, which means signatures are very large.
To get shorter signatures it is natural to try to increase the size of the challenges.
In this section we sketch an approach to obtain $s$-bit challenge values for any $s \in \N$, by trading the challenge size with the public key size. In the next section we explain how to shorten the public keys again.


The basic idea is to have public keys $( E_{A,1} = \a_1 * E , \dots , E_{A,2^s} = \a_{2^s} * E )$.
For each $1 \le m \le 2^s$ we choose $\e_m \leftarrow [-B,B]^n$ and set $E_{A,m} = ( \prod_{i=1}^n \l_i^{e_{m,i}} ) * E$.
The signing algorithm for user A chooses $t$ random ideals $\b_k = \prod_{i=1}^n \l_i^{f_{k,i}}$ and computes $E_{A,k} = \b_k * E$, as before.
Now we have $s$-bit challenges $b_1, \dots, b_t \in \{1, 2, \dots, 2^s \}$.
For each $1 \le k \le t$ the signer computes $\z_k = \f_k - \e_{b_k}$, which corresponds to the ideal class $\c_k := \a_{b_k}^{-1} \b_k$ and the verifier can check that $j( E_k ) = j( \c_k * E_{A, b_k})$.

For security we now only require $ts \ge \lambda$. Taking, say, $\lambda = 128$ and $s = 16$ can mean $t$ as low as 8, and so only 8 ideal classes need to be transmitted as part of the signature.
Of course the public key now includes $2^{16}$ $j$-invariants (elements of $\F_p$) which would be around 4 megabytes.


As far as we can tell, this idea cannot be applied to the schemes of Yoo et al~\cite{YAJJS17} or Galbraith et al~\cite{GPS17}.



\subsection{Security}

A trivial modifucation to the proof of Theorem~\ref{thm:security-basic} can be applied in this setting. But note that the forking lemma produces two signatures such that $b_k \ne b_k'$ for some index $k$.
Hence from a successful forger we derive two ideal classes $\c_k$ and $\c_k'$ such that $j( \c_k * E_{A, b_k} ) = j( \c_k' * E_{A, b_k'})$. It follows that $(\c_k')^{-1} \c_k$ is an ideal class corresponding to an isogeny $E_{A,b_k} \to E_{A,b_k'}$.
Hence the computational assumption underlying the scheme is the following.

\begin{definition}\label{defn:one-out-of-2s-problem}
(One out of $2^s$ ideal action problem ***)
Let notation be as in the key generation protocol of the scheme.
Consider a set of $2^s$ elliptic curves $\{ E_{A,1}, \dots, E_{A,2^s} \}$, all of the form $E_{A,m} = \a_m * E$ for some ideal $\a_m = \prod_{i=1}^n \l_i^{e_{m,i}}$ where the exponent vectors $\e_m $ are uniformly sampled in $[-B,B]^n \subseteq \Z^n$. The ``one out of $2^s$ ideal action problem'' is to compute an ideal corresponding to any isogeny $E_{A,m} \to E_{A,m'}$ for some $m \ne m'$.
\end{definition}


*** Is this easier than the general problem?  Don't have a reduction, but.....


\begin{theorem}
In the random oracle model, the signature scheme of this section is unforgeable under a chosen message attack under the assumption of Definition~\ref{defn:one-out-of-2s-problem}.
\end{theorem}

The proof of this theorem is almost identical to the proof of Theorem~\ref{thm:security-basic} and so is omitted.



\subsection{Variant based on a more natural problem}


Definition~\ref{defn:one-out-of-2s-problem}  is a little un-natural.
It would be more pleasing to prove security based on the problems in Definition~\ref{defn:ass1} or~\ref{defn:ass1p}.
We now explain that one can prove security based on the problem in Definition~\ref{defn:ass1}, under an assumption about uniform sampling of ideal classes.

So suppose in this section that the distribution $\D_B$ of Definition~\ref{defn:sampling-distributions} has negligible statistical distance from the uniform distribution.
This assumption is reasonable for bounded $n$ and very large $B$; but we leave for future work to determine whether practical parameters for isogeny crypto can be obtained under this constraint.


We add an extra $1/2$ factor in the success probability, but this is not a big issue since the security proof isn't tight anyway.


\begin{theorem}
Let parameters be such that the statistical distance between $\D_B$ and the uniform distribution on $\Cl(\OO)$ is negligible.
In the random oracle model, the signature scheme of this section is unforgeable under a chosen message attack under the assumption of Definition~\ref{defn:ass1}.
\end{theorem}


(***) Just noticed notation clash $E_A$ for public key of A = Alice; A = adversary against scheme. Could use $\FF$ for forger?  Probably doesn't matter.

\begin{proof}
Let $A$ be an adversary against the scheme and $(E, E_A = \a * E )$ be an instance of Definition~\ref{defn:ass1}.

Choose random ideal classes $\b_1, \dots, \b_{2^s}$
(chosen as $\b_m = \prod_{i=1}^n \l_i^{u_{i,m}}$ for $1 \le m \le 2^s$ and $u_{i,m} \in [-B,B]$)
and compute $E_{A,m}' = \b_m * E$ for $1 \le m \le 2^{s-1}$ and $E_{A,m}' = \b_m * E_A$ for $2^{s-1} < m \le 2^s$. Choose a random permutation $\pi$ on $\{ 1, 2, \dots, 2^s \}$ and set the public key to be the sequence $E_{A,m} = E_{A,\pi(m)}'$.
Note that these curves are all uniformly sampled in the isogeny class, and so there is no way to distinguish whether the curve has been generated from $E$ or $E_A$.
Now run the adversary on this public key.

This is where the subtlety about distributions appears: it is crucial that the public key derived from the pair $(E, E_A)$ is indistinguishable from public keys output by the key generation algorithm.


The simulator will handle hash and sign queries similarly to the proof of Theorem~\ref{thm:security-basic}.
When simulating the sign oracle we first choose the $t$ values $b_1, \dots, b_1 \in \{0,1\}^s$, which are interpreted as the challenge values $1 \le b_k \le 2^s$.
For each $1 \le k \le t$ we choose a random ideal class $\c_k = \prod_{i=1}^n \l_i^{z_{k,i}}$ by choosing $z_{k,i} \in [-ntB, ntB]$, and then compute $E_k = c_k * E_{A,b_k}$.
One then defines the hash value $H( j(E_1), \dots, j(E_t), \text{message} )$ to be $b_1 \Vert b_2 \Vert \cdots \Vert b_t$.
This simulation is indistinguishable to the real cryptosystem, assuming the distributions on ideal classes have negligible statistical distance.

When the forking lemma is applied we have, for some index $k$, two different ideals $\c_k, \c_k'$ such that $j( \c_k * E_{A, b_k} ) = j( \c_k' * E_{A, b_k'})$ and $b_k' \ne b_k$
With probability $1/2$ we have that one of the $E_{A,b_k}$ is known to the simulator as  $\b * E$ and the other is known as $\b' * E_A$. If this event occurs then we have $\c_k \b * E = \c_k' \b' * E_A$ (or vice versa) in which case $(c_k' \b')^{-1} c_k \b$ is a solution to the original instance.
\end{proof}


%Version 2: Have a computational assumption $(E, \a * E, \a^2 * E, \a^3 * E , \dots, a^{2^s - 1} * E )$ and 3-special soundness implies can get $\a$ with a good probability. This is cute but ultimately pointless.


\subsection{Reducing storage for private keys}\label{sec:private-key-compress}

Rather than storing all the private keys $\a_i$ one could have generated them using a pseudorandom generator as $PRG( \seed, i )$ where $\seed$ is a seed and $i$ is used to generate the $i$-th private key (which is an integer exponent vector).
The prover only needs to store $\seed$ and can then recompute the private keys as needed.
Of course, during key generation one needs to compute all the public keys, but during signing one only needs to determine $t \approx 8$ private keys (although this adds a cost to the signing algorithm).





\section{Smaller public keys} \label{sec:smaller-keys}

The approach of Section~\ref{sec:smaller-sigs} gives signatures that are potentially quite small, but at the expense of very large public keys. In some settings (e.g., software signing) large public keys can be easily accommodated, while in other settings (e.g., certificate chains) it makes no sense to shorten signatures at the expense of public key size.
In this section we explain how to use a Merkle trees to compress the public key while also maintaining compact signatures.
The public key is now just a hash value.

Let $E_{A,1}, \dots, E_{A,2^s}$ be the curves in the public key.
Recall from Section~\ref{sec:private-key-compress} that the public keys can be recomputed on-the-fly if needed.


We set up a hash tree by defining $h_{l,u}$ for $0 \le l \le s$ and $1 \le u \le 2^{s-l}$.
First set $h_{0,u} = j( E_{A,u} )$ for $1 \le u \le 2^s$.
%The first row is defined as
%\[
%   h_{1,1} = H( j( E_{A,1}) , j(E_{A,2} ), \dots,  h_{1,u} = H( j( E_{A,2u-1}), j( E_{A, 2u} ) ),
% \dots,  h_{1,2^{s-1}} = H( j( E_{A,2^s-1}), j( E_{A, 2^s} ) ).
%\]
Now, for any $1 \le l \le s$, the rows of the hash tree are defined as
\[
   h_{l,u} = H( h_{l-1,2u-1}, h_{l-1,2u} ).
\]
The public key is the final value $h_{s,1}$.


A signature now needs to contain additional information.
The signature contains ideals $\c_1, \dots, \c_t$ that relate to public keys $E_{A,b_1}, \dots, E_{A,b_t}$. It is necessary for the signature to now contain these $t$ curves (represented using $j$-invariants) and also enough of the hash values $h_{l,u}$ to allow the verifier to confirm that these curves correspond to leaves of the Merkle tree.
Hence the signature also includes up to $t$ values $h_{0,u}$ that correspond to ``neighbours'' of $E_{A,b_1}$.
Then, as usual, we provide the hash values coming off the path to the root.

In the worst case (????)
it is necessary to send around $ts$ hash values as part of the signature.
(** TO DO: Work out exactly **)
For our parameters $t=8, s=16$ and 128-bit hash values, this adds about 16384 bits to the signature (2kb).

(*** I don't understand how you got such a small value in the below table ***)


NOTE: Boneh tells me Merkle-tree approaches are asymptotically not very beneficial, so we should put some thought into that. But it seems to have benefits in practice for certain security levels anyway.



Note: This may seem close to hash-based signatures and so one might ask why to use isogenies rather than just using hash-based signatures. The answer is that our scheme is stateless and does not leak information over time. It is just a compression technique.



\section{Conclusion}


The below table gives some estimates of cost.
Some of the parameter constraints may not be necessary.


QUESTION: I guess the costs of sign/verify time are really the cost of computing the isogeny of that size. Sign time is potentially worse due to the rejection sampling. We need to compute a bunch of $f_{k,i}$ and then compute the isogenies and then possibly throw it all away and start again. (!!!!!!)


\begin{table}
  \centering
  \begin{tabular}{l | c | c | c | c |}
    & Basic scheme
    & Reject. sampl. (Section~\ref{sec:basic-scheme})
    & Parallel proof (Section~\ref{sec:smaller-sigs})
    & Compressed public keys (Section~\ref{sec:smaller-keys}) \\
    \hline
    \hspace{1em}\textbf{Exact} &&&&\\
    Sig size
    & $\lambda\lceil n\log (2B+1)\rceil + \lambda$
    & $\lambda\lceil n\log (2n\lambda B + 1)\rceil + \lambda$
    & $\frac{\lambda}{s}\lceil n\log (2n\frac{\lambda}{s}B + 1)\rceil + \lambda$
    & $\frac{\lambda}{s}(\lceil n\log (2n\frac{\lambda}{s}B + 1)\rceil + \log p) + (s+1)\lambda$\\
    PK size
    & $\log p$ & $\log p$ & $2^s\log p$ & $\lambda$ \\
    SK size
    & $n\log(2B+1)$ & $n\log(2B+1)$ & $2^s n\log(2B+1)$ & $(2^s+1) \lambda$\\
    $\Cl(\OO)$ actions &&&&\\
    $\to$ keygen
    & 1 & 1 & $2^s$ & $2^s$\\
    $\to$ sig/verify
    & $\lambda$ & $n\lambda^2$ & $n(\lambda/s)^2$ & $n(\lambda/s)^2$\\
    \hline
    \hspace{1em}\textbf{Asymptotic} &&&&\\
    Sig size
    & $O(\lambda^3)$ & $O(\lambda^3)$ & $O(\lambda^3/s)$ & $O(\lambda^3/s)$\\
    PK size
    & $2\lambda^2$ & $2\lambda^2$ & $2^{s+1}\lambda^2$ & $\lambda$\\
    SK size
    & $\lambda^2$ & $\lambda^2$ & $2^s\lambda^2$ & $(2^s+1)\lambda$\\
    $\Cl(\OO)$ actions &&&&\\
    $\to$ keygen
    & 1 & 1 & $2^s$ & $2^s$\\
    $\to$ sig/verify
    & $\lambda$ & $\lambda^4/\log(\lambda)$
    & $(\lambda^2/s)^2/\log(\lambda)$ & $(\lambda^2/s)^2/\log(\lambda)$\\
    \hline
    \hspace{1em}\textbf{CSIDH(128,16)} &&&&\\
    Sig size
    & 4112 B & 19600 B & 944 B & 1209 B\\
    PK size
    & 63 B & 63 B & 4032 KB & 16 B\\
    SK size
    & 32 B & 32 B & 2048 KB & 1024.02 KB \\
    Est. sig/verify time
    & 13 s & 122000 s & 474 s & 474 s\\
    Est. keygen time
    & 0.1 s & 0.1 s & 6554 s & 6554 s
  \end{tabular}
  \caption{Parameter size and performance of the various signature
    protocols.
    Equivalences used for asymptotic analysis are:
    $\log p \sim 2\lambda^2$, $n\log B\sim \lambda^2$,
    $n\log n \sim 2\lambda^2$.
    All logarithms are in base 2. % NdL: this is probably wrong
  }
  \label{tab:comparison}
\end{table}


\section{Conclusions}


We have given a signature scheme suitable for the CSIDH isogeny setting.
This solves an unresolved problem in Stolbunov's thesis.
We have also shown how to get shorter signatures by increasing the public key size.
It seems that a similar tradeoff between public key size and signature size cannot be done in the schemes of Yoo et al or Galbraith et al, based on the SIDH setting.


\bibliographystyle{plain}
\bibliography{biblio}



\appendix


\section{Variants}

One can consider various ideas to get more efficient (i.e., faster signing and verification) or more compact signatures.

\begin{enumerate}
\item Following Stolbunov one could use higher powers for the smaller primes.

This is definitely worth looking at. We could take $B=1$ for some of the larger primes (e.g., the ones bigger than 50), and then use much larger values for $B$ for the primes $3, 5, 7$ etc. If $B_i$ is the bound used for $\l_i$ then the full key space is $\prod_i (2B_i + 1)$.

The main problem is that a change to a single $B_i$ makes very little difference to the key size. For example, doubling $B_1$ only adds roughly $+1$ to the logarithm of the product, which is the security parameter.
Hence the main factor in having a large security parameter is using lots of distinct primes $\l_i$, which automatically means a high cost for signing and verification since almost all $f_{k,i}$ will be non-zero.



\item Could sample the exponents $e_i$ from a discrete Gaussian distribution (or perhaps some other distribution) as has been done with lattice signatures.

Suppose the $e_i$ are sampled from a discrete Gaussian distribution with parameter $\sigma$, so that the standard deviation is close to $\sigma$. The entropy of the continuous Gaussian distribution with standard deviation $\sigma$ is $\log( 2 \pi e \sigma^2 )/2$.
% so we need the $n$-th multiple of this to be larger than the security parameter.

For example, take $n=70$, $s = 16, t = 8$ and choose $\sigma = 5$ (so almost all values $e_{i}$ will lie in $[-15,15]$ but occasionally one is larger than this. Then
\[
   n \log( 2 \pi e \sigma^2 )/2 \approx 212
\]
so determining the $e_i$ using a meet-in-the-middle strategy should require at least $2^{106}$ iterations, and realistically much more than this since organising a search based on the entropy is hard to do. For post-quantum security we might want to replace $2 \lambda$ with $3\lambda$ in the above.

We following the methods and results of Lyubashevsky~\cite{Lyu12}. Lemma 4.3(3) of~\cite{Lyu12} shows we can bound the norm $\Vert \e \Vert$ by $T = 2 \sigma \sqrt{n}$.
Now we need to choose the $f_{k,i}$ from a discrete Gaussian with parameter $\sigma'$, so that the distribution of $f_{k,i} + e_i$ is close (within statistical distance, though we could use Renyi divergence to get better results) to the discrete Gaussian with parameter $\sigma'$.
Lemma 4.6 of~\cite{Lyu12} suggests that $\sigma' = T \sqrt{\log(n)}$ is sufficient. Since we need to repeat this $t$ times we suggest $\sigma' = T \sqrt{\log(nt)}$.

For our choices $n=70, t = 8, \sigma = 5$ this gives $\sigma' \approx 210$.
If we use some kind of compact coding of integers distributed as Gaussians~\cite{DDLL13} then signature size would be at best $nt \log( 2 \pi e (\sigma')^2 )/2$ bits.
For our example parameters this would be 3790 bits, or around 474 bytes.

More work needed to work out the exact details and determine the rejection sampling probabilities. ***

\end{enumerate}


\section{Tight security reduction based on lossy keys}

I've had a quick look at: Eike Kiltz, Vadim Lyubashevsky, Christian Schaffner (A Concrete Treatment of Fiat-Shamir Signatures in the Quantum Random-Oracle Model. EUROCRYPT (3) 2018: 552-586)

I think we can use these ideas. Here's the way to get a lossy scheme:
Take a very large class group, but use a small $n$, $B$ so that $\{ \a = \prod_{i=1}^n \l_i^{e_i} : e_i \in [-B,B] \}$ is a very small subset of the class group.\footnote{Might even be able to consider working with subgroups, in the quantum case where the class group structure is known.}
The real key is $(E, E_A = \a*E )$ for such an $\a$.
The lossy key is $(E, E_A )$ where $E_A$ is a uniformly random curve in the isogeny class.
Further, chooose parameters so that the $f_{k,i}$ are also such that $\{ \b = \prod_{i=1}^n \l_i^{f_{k,i}} : |f_{k,i}| \le (nt+1)B \}$ is a small subset.
In the case of a real key, the signatures include ideals that correspond to paths from $E$ or $E_A$ to a  curve.
In the case of a lossy key, then such ideals do not exist as for a curve $E'$ it is not the case that there is a short path from $E$ to $E'$ AND a short path from $E_A$ to $E'$.
But we can simulate the signatures in the ROM by choosing $E'$ appropriately.

I'll need to check the details of this paper to see if this works out.



\section{Using the relation lattice}\label{sec:sig-relation-lattice}

This section explains an alternative solution to the problem of representing an ideal class without leaking the private key of the signature scheme.
This variant makes sense if a quantum computer is available during system setup.


Let $\{ \l_1, \dots, \l_n \}$ be a set of $\OO$-ideals that generates $\Cl( \OO )$.
Define $L = \{ (x_1, \dots, x_n ) \in \Z^n : \prod_{i=1}^n \l_i^{x_i} \cong (1) \}$.
Then $L$ is a rank $n$ lattice with volume equal to $\#\Cl(\OO)$.
We call this the \emph{relation lattice}.

A basis for this lattice can be constructed in subexponential time using classical algorithms~\cite{hafner1989rigorous,biasse_fieker_jacobson_2016} or in probabilistic polynomial time using quantum algorithms (define $f:\Z^n\to\Cl(\OO)$ by $f(x_1,\dots,x_n)=\prod_{i=1}^n\l_i^{x_i}$, then $f$ can be evaluated in polynomial time~\cite{shanks1989gauss,Cohen1993}, and finding a basis for $L=\ker f$ is an instance of the Hidden Subgroup Problem for $\Z^n$, which can be solved in polynomial time using Kitaev's generalization of Shor's algorithm~\cite{kitaev1995hsp}).
The classical approach is not very interesting since the underlying computational assumption is only subexponentially hard for quantum computers, but it might make sense in a certain setting.
The quantum case would make sense in a post-quantum world where a quantum computer can be used to set up the system parameters for the system and then is not required for further use.
It might also be possible to construct $(E, p )$ such that computing the relation lattice is efficent (e.g., constructing $E$ so that $\Cl( \End(E))$ has smooth order), but we do not consider such approaches in this paper.

For the remainder of this section we assume that the relation lattice is known.
Let $\{ \x_1, \dots, \x_n \}$ be a basis for $L$
Let $\FF = \{ \sum_{i=1}^n : u_i \x_i : -1/2 \le u_i < 1/2 \}$ be the centered fundamental domain of the basis of $L$.
Then there is a one-to-one correspondence between $\FF \cap \Z^n$ and $\Cl(\OO)$ by
$(z_1, \dots, z_n ) \in \FF \cap \Z^n  \mapsto \prod_{i=1}^n \l_i^{z_i}$.
In practice one prefers a basis for $L$ so that all vectors in $\FF$ have relatively short norm, which is achieved by taking the basis to be as short and close to orthogonal as possible. Hence one applies lattice basis reduction to obtain as ``nice'' a basis for $L$ as possible.

Note that, given a basis $\{ \x_1, \dots, \x_n \}$ for $L$ and a vector $\z = (z_1, \dots, z_n ) \in \Z^n$ one can efficiently compute the unique vector in $\FF \cap (\z + L )$ using the Babai rounding method~\cite{Bab86}.



Returning to Stolbunov's signature scheme, the solution to the problem is then straightforward:
Given $\a = \prod_{i=1}^n \l_i^{e_i}$ and $\b_k = \prod_{i=1}^n \l_i^{f_{k,i}}$,
a representation of $\b_k \a^{-1}$ is obtained by computing the vector $\z' = \f_k - \e = (f_{k,i} - e_k)$
and then using Babai rounding to get the unique vector $\z$ in $\FF \cap (\z' + L )$.
The vector $\z$ is sent as the response to the $k$-th challenge.
Since $\b_k$ is a uniformly chosen ideal class, the class $\b_k \a^{-1}$ is also uniformly distributed as an ideal class, and hence the vector $\z \in \FF \cap \Z^n$ is uniformly distributed and carries no information about the private key.

\begin{lemma}
If $\b_k$ is a uniformly chosen ideal class then the vector 
$\z \in \FF \cap \Z^n$ corresponding to $\f_k - \e$ is uniformly distributed.
\end{lemma}

\begin{proof}
For fixed $\e$ the vector $\z$ depends only on the ideal class of $\b_k$.
But $\b_k$ is uniform and independent of $\e$ and not known to verifier.
\end{proof}



If the basis for $L$ is sufficiently nice then one can obtain good bounds on the size of the vectors $\z$.

One final remark: In the security proof we need to be able to simulate the signing oracle, and hence we need to produce uniformly chosen vectors $\z \in \FF \cap \Z^n$.
The simplest way to do this is to uniformly sample $\z'$ in a large box in $\Z^n$ and then apply Babai rounding as above.
Proving anything about this seems to be hard: How large is the box? How close to uniform?
This ended up in the ``too hard'' pile.


\begin{lemma} \label{lem:sim1}
Let $B \in \N$. Let $D_1$ be the distribution on ideal classes obtained by computing $\prod_{i=1}^n \l_i^{x_i}$ over uniformly sampled $x_i \in [-B,B]$.
Suppose the statistical distance between $D_1$ and the uniform distribution on $\Cl(\OO)$ is bounded by $\epsilon$.
Let $D_2$ be the distribution on $\FF \cap \Z^n$ defined by uniformly sampling vectors $\x \in [-B,B]^n$ and applying Babai rounding.
Let $U$ be the uniform distribution on $\FF \cap \Z^n$.
Then the statistical distance between $D_2$ and $U$ is at most $\epsilon$.
\end{lemma}

Note $[-B,B]$ is the set of integers $u$ with $-B \le u \le B$.


\begin{proof}
They are the same thing.
\end{proof}


One can then prove a variant of Theorem~\ref{thm:security-basic} and all the theorems in the paper. This approach should give rise to much smaller signatures -- close to optimal size given the subexponential security of the class group action problem.

\end{document}





